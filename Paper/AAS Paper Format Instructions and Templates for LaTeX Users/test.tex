% use paper, or submit
% use 11 pt (preferred), 12 pt, or 10 pt only

\documentclass[letterpaper, preprint, paper,11pt]{AAS}	% for preprint proceedings
%\documentclass[letterpaper, paper,11pt]{AAS}		% for final proceedings (20-page limit)
%\documentclass[letterpaper, paper,12pt]{AAS}		% for final proceedings (20-page limit)
%\documentclass[letterpaper, paper,10pt]{AAS}		% for final proceedings (20-page limit)
%\documentclass[letterpaper, submit]{AAS}			% to submit to JAS

\usepackage{bm}
\usepackage{amsmath}
\usepackage{subfigure}
%\usepackage[notref,notcite]{showkeys}  % use this to temporarily show labels
\usepackage[colorlinks=true, pdfstartview=FitV, linkcolor=black, citecolor= black, urlcolor= black]{hyperref}
\usepackage{overcite}
\usepackage{footnpag}			      	% make footnote symbols restart on each page




\PaperNumber{XX-XXX}



\begin{document}
	
	\title{MANUSCRIPT TITLE (UP TO 6 INCHES IN WIDTH AND CENTERED, 14 POINT BOLD FONT, MAJUSCULE)}
	
	\author{Junette Hsin\thanks{Title, department, affiliation, postal address.},  
		Daniel Gonzalez\thanks{Title, department, affiliation, postal address.},
		\ and J.Q. Public\thanks{Title, department, affiliation, postal address.}
	}
	
	
	\maketitle{} 		
	
	
	\begin{abstract}
		
		The abstract should briefly state the purpose of the manuscript, the problem to be addressed, the approach taken, and the nature of results or conclusions that can be expected. It should stand independently and tell enough about the manuscript to permit the reader to decide whether the subject is of specific interest. The abstract shall be typed single space, justified, centered, and with a column width of 4.5 inches. The abstract is not preceded by a heading of ``Abstract'' and its length may not extend beyond the first page.
		
	\end{abstract}
	
	
	
	
\end{document}
